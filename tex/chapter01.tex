%# -*- coding: utf-8-unix -*-
%%==================================================
%% chapter01.tex for SJTU Bachelor Thesis
%%==================================================

%\bibliographystyle{sjtu2}%[此处用于每章都生产参考文献]
\chapter{绪论}
\label{chap:Intro}
%\sunxuezhang{建议。。。}
\section{课题研究背景}
在第一代与第二代无线移动通信中,蜂窝网络(cellular)或无线电接入网(Radio Access Networks,缩写RAN)由相互独立的基站组成,每个基站配有电力、备用电源、空调系统、环境监测、回程链路等其他支持设施。每个基站内部含有射频信号发射单元,射频信号通过电缆从地面机柜传输至位于高处的基站天线。为了降低射频信号在电缆上的传输损耗,第三代移动通信采用了分布式基站系统。分布式基站将射频拉远单元(Remote Radio Head,缩写RRH)与基带信号单元(Baseband Unit,缩写BBU)分离,使RRH可以直接部署在基站塔顶距离天线很近的地方,并通过光纤的连接使两者间的距离变大(可达几百米或几千米)而损耗减小。建立在分布式基站架构的基础上,Cloud RAN为无线接入网引入了云技术,使很多BBU集中到一个BBU池。通过应用最近的数据中心网络技术,Cloud RAN能够在BBU池中的BBU之间建立低功耗、高可靠、低延迟、高带宽的连接。它应用了开放的平台与实时虚拟化技术以实现动态共享的资源分配并支持多厂商、多种技术并存的环境。在这种技术架构下,由于BBU池覆盖的范围比原来的一个基站所能覆盖的范围大得多(即一个用户从一个RRH切换到另一个RRH时,他前后连接的两个BBU在同一个BBU池中)\cite{ran2015optimal},蜂窝系统中流量分布不均匀的问题可以通过在BBU池中部署灵活的、动态自适应的负载均衡算法得以解决。

\section{国内外研究现状}
传统蜂窝网络中的负载均衡算法已被大量研究,如小区呼吸技术\cite{niu2010cell}已被广泛应用于第二代和第三代移动通信网络中。由X. Lin和S. Wang提出的Cloud RAN中的高效RRH切换机制在考虑系统能效条件下实现了负载均衡\cite{lin2010efficient}。C. Ran和S. Wang等人\cite{ran2015optimal}从另一个角度提出了Cloud RAN中的最佳负载均衡算法。他们周期性地监控衡量均衡度的公平指数,当其低于特定阈值时,则重新设计每个小区覆盖的区域以实现系统的负载均衡。C. Tsai和M. Moh在Cloud RAN架构下比较了8种均以降低物联网(Internet of Things,简称IoT)设备通信延迟为目标的负载均衡算法。这些算法或者采用预先定义的指标与阈值以实现监测与负载均衡,或者采用了较为固定而非实时改变的策略。与之相反,B. Shahriari和M. Moh在部分可见的动态环境的情景下提出的一般线上学习(Generic Online Learning,简称GOL)系统可以很好地应用于实时解决5G Cloud RAN中的负载均衡问题。但是,两人的工作主要是针对数据中心网络中虚拟机的高速缓存与云存储上的负载均衡内容,与本课题主要关注的数据流的选路及拥塞信息采集不同。现阶段主流的数据中心负载均衡算法分为两类,一类是集中式负载均衡(如Hedera,Fastpass等),拥塞信息采集的工作由中心化的操作完成,这一类算法可以关注到全局的数据特点,更好地进行选路,但是一般而言其信息流的延迟较高;另一类算法是分布式负载均衡算法(如CONGA,CLOVE等),它们利用本地的信息使交换机作出选路等决策,有较低延迟,但对于全局信息的掌握不像集中式那样清晰。与负载均衡减少数据流完成时间的目标相同,交换机处缓存的流量调度机制也被广泛研究,Wei Bai等人提出的PIAS架构利用多优先级反馈队列,对数据流的优先级进行细化标记,提升了流量调度的性能。一般而言,目前的负载均衡算法与流量调度算法在研究时均默认两者中的另一方是性能优良的算法,本文将二者联合起来考虑,构建一个综合系统,以实现更好的流量优化。


\section{本文研究思路及组织结构}
在B. Shahriari和M. Moh工作的启发下,本课题旨在采用深度增强学习的方法提升Cloud RAN架构中负载均衡算法的性能。通过利用深度学习可以从环境与智能系统交互及历史信息等中提取抽象特征的能力,并借助增强学习系统感知环境、接收奖励与惩罚的能力,构建增强学习算法解决均衡问题。本论文的第二章讨论增强学习算法,从基于值函数的增强学习、基于策略的增强学习、actor-critic算法三方面介绍深度神经网络与三种增强学习算法的结合;第三章讨论5G Cloud RAN中的负载均衡问题,此处的负载均衡是指广义的负载均衡,即包含了从端系统到交换机到端系统整体路径上的负载均衡,涵盖数据流的优先级标记、交换机缓存处的入队、出队操作及选路、拥塞信息采集等一系列过程,在对5G时代背景下的数据中心有了基本认识之后讨论目前的负载均衡及流量调度算法,如MPTCP、CONGA、Expeditus、LocalFlow、DRILL、PIAS等;第四章介绍本文提出的基于深度增强学习的流量优化系统,从系统架构和仿真过程两方面阐述系统的实现、仿真环境、性能分析等内容;最后总结全文,并对基于本文的进一步的研究提出一些见解。